\documentclass{beamer}

\usepackage{amssymb}
\usepackage{amsmath}
\usepackage{mathabx}
\usepackage[french]{babel}
\usepackage[utf8]{inputenc}
\usepackage[T1]{fontenc}
\usepackage[]{eufrak}
\usepackage[]{verbatim}
\usepackage{graphicx}
\graphicspath{{images/}}
\usepackage{float}

\renewcommand{\FrenchLabelItem}{\textbullet}

\usetheme{Warsaw}
\title[Programmation Réseau]{Projet de Programmation Réseau}
\author[Emile CONTAL]{Charles-Pierre Astolfi, Raphael Bonaque, Emile Contal, Martin Gleize, David Montoya}
\date{13 Janvier 2011}

\begin{document}

\begin{frame}
  \titlepage
  \begin{figure}[H]
    \begin{center}
      \includegraphics[scale=0.05]{logo.jpg}
    \end{center}
  \end{figure}
\end{frame}


\section{Un aperçu du programme}
\subsection{Connexion au serveur}
\begin{frame}
  \begin{figure}[H]
    \begin{center}
      \includegraphics[scale=0.35]{screen1.jpg}
    \end{center}
  \end{figure}
\end{frame}

\begin{frame}
  \begin{figure}[H]
    \begin{center}
      \includegraphics[scale=0.35]{screen2.jpg}
    \end{center}
  \end{figure}
\end{frame}

\subsection{Les parties en attente}
\begin{frame}
  \begin{figure}[H]
    \begin{center}
      \includegraphics[scale=0.35]{screen3.jpg}
    \end{center}
  \end{figure}
\end{frame}

\begin{frame}
  \begin{figure}[H]
    \begin{center}
      \includegraphics[scale=0.35]{screen4.jpg}
    \end{center}
  \end{figure}
\end{frame}

\subsection{Le Tetris}
\begin{frame}
  \begin{figure}[H]
    \begin{center}
      \includegraphics[scale=0.35]{screen6.jpg}
    \end{center}
  \end{figure}
\end{frame}

\begin{frame}
  \begin{figure}[H]
    \begin{center}
      \includegraphics[scale=0.35]{screen5.jpg}
    \end{center}
  \end{figure}
\end{frame}

\section{Le Protocole}
\subsection{Le protocole en deux mots}
\begin{frame}

\begin{block}{Le protocole en deux mots}
\begin{itemize}
\item TCP
\item Spécification octet par octet
\end{itemize}
\end{block}

\begin{block}{Conventions d'écriture}
\begin{itemize}
\item Big-endianness
\item entiers non signés
\item description précise de chaque nouvel objet (clés RSA et DSA,
 actions du round, etc...)
\end{itemize}
\end{block}

\end{frame}

\section{Bilan}
\subsection{Ce qu'il me resterait à faire}
\begin{frame}
\begin{block}{Ce qu'il me resterait à faire}
\begin{itemize}
\item Déconnexion des clients
\item Base de donnée partagée
\item Détection des tricheurs
\end{itemize}
\end{block}
\end{frame}

\subsection{Conclusion}
\begin{frame}
\begin{block}{Conclusion}
\begin{itemize}
\item Travail en groupe (réunion, chat vocal, spécifications communes,
  ...)
\item Confrontation au bas niveau des langages et du réseau
\end{itemize}
\end{block}
\end{frame}


\end{document}